% !TEX root = ../document.tex

\chapter{Erstellung der Versuchsumgebung}

\cleanchapterquote{Wenn Sie der Meinung sind, dass gutes Design teuer ist, sollten Sie sich die Kosten für schlechtes Design ansehen.}{Ralf Speth}{Ehem. Geschäftsführer bei Jaguar Land Rover}

\section{Die \name{index.html}}

Die Versuche sind jeweils eigenständige HTML-Seiten, welche über ein HTML-Iframe Element in einer zentralen \name{index.html} untersucht werden können.
Die Versuche sind wiederum in einzelne Sachverhalte gruppiert.
Für diese Gruppierung wird in der \name{index.html} ein \lstinline{tests} Objekt definiert.

\begin{lstlisting}[language=JavaScript, caption={Ausschnitt der Definition des \lstinline{tests} Objekts in der \name{index.html}.}, label={lst:tests_objekt}]
const tests = {
    pendel: {
        title: "Drehpunkt bei gegebenem Glied.",
        test: {
            "kleinster umfassender Kreis": [
                "Erster Versuch",
                "Filtern bekannter Punkte",
                "Pfad des Mittelpunktes",
                "Drehpunkt via Orthogonale",
            ],
            "schnittpunkte": [
                "Weitest entfernte Punkte",
                "5 weiteste Punkte",
            // ...
\end{lstlisting}

Über die Schlüssel des \lstinline{tests} Objekts werden die Gruppen festgehalten.
Die Werte von \lstinline{tests} definieren die in dieser Gruppe gemachten Versuche.
Es sind entsprechend Listen von Titeln für die Versuche darin enthalten.

Die automatische Einbindung der Versuchsdateien geschieht über die analoge Ordnerstruktur des Projektes.
Jede Eigenschaft des \lstinline{tests} Objekts hat im \name{src} Ordner relativ zum Projektverzeichnis einen genauso bezeichneten Ordner.
Dieser enthält die einzelnen HTML-Versuchsseiten, welche ebenfalls nach einem festen Schema benannt sind.
Die Versuchsgruppen sind \name{pendel}, \name{opticalflow}, \name{momentanpol}, \name{gruppen} und \name{bilder}.
Der in Abbildung~\ref{fig:index.html} gezeigte Test ist entsprechend unter dem Relativpfad \name{src/pendel/pendel1\_4.html} zu finden.
Anhand dieser Strukturierung werden alle Versuche durch den in Listing~\ref{lst:tests_objekt_reason} beschriebenen Aufruf dynamisch in die \name{index.html} eingebunden.

\begin{lstlisting}[language=JavaScript, caption={Iteration über das tests Objekt zur Befüllung der Navigationsleiste.}, label={lst:tests_objekt_reason}]
Object.entries(tests).forEach((kv, i) => {
    const title = create('h2', null, kv[1].title);
    const details = create('details', byId('sidenav'));
    const summary = create('summary', details, kv[1].title);

    Object.entries(kv[1].test).forEach((e, j) => {
        const innerDetails = create('details', details);
        const innerSummary = create('summary', innerDetails);
        innerDetails.classList.add('innerDetails');
        innerSummary.innerHTML = e[0];

        const ul = create('ul', innerDetails);

        for (let m = 0; m < e[1].length; ++m) {
            const name = `${kv[0] + (j + 1)}_${m + 1}.html`;
            const li = create('li', ul, e[1][m] || name);
            li.addEventListener('click', () => {
                byId('iframe').src = `src/${kv[0]}/${name}`;
                Array.from(document.getElementsByTagName('li'))
                    .forEach(e => e.style.listStyle = 'disc');
                li.style.listStyle = 'inside';
            });
        }
    });
});
\end{lstlisting}

\begin{figure}
    \includegraphics[width=\textwidth]{gfx/index.png}
    \caption[Bild der \name{index.html}]{Bild der \name{index.html}, wobei in der Navigationsleiste der Versuch zur Ermittlung des Drehpunktes ausgewählt ist. Der Hauptteil der Seite wird durch das HTML-Iframe Element gefüllt, welches den entsprechenden Versuch enthält.}\label{fig:index.html}
\end{figure}

Hier wird für jeden Eintrag im \lstinline{tests} Objekt ein neues HTML-Details Element erstellt.
Über die Listen der einzelnen Einträge wird iteriert, um die einzelnen Gruppen gemeinsam in ein weiteres HTML-Details Element zu platzieren\footnote{Hier als \lstinline{innerDetails} bezeichnet.}.
Diese Listen werden durch HTML-Listen Elemente gerendert.
Für die Einträge der Listen werden entsprechende \lstinline{EventListener}\footnote{Für mehr Information über \lstinline{EventListener} siehe \aka{https://developer.mozilla.org/en-US/docs/web/api/eventlistener}.} registriert, welche die \lstinline{src} Eigenschaft des HTML-Iframe Elements aktualisieren.
Sollte der Pfad einer Datei nicht diesem Aufbau entsprechen, so wird ein Fehler geworfen.

Für die Formatierung der \name{index.html} wurden die entsprechenden Elemente mit Hilfe von CSS gestaltet, worauf hier jedoch nicht weiter eingegangen werden soll.
Der Quelltext kann unter \aka{https://github.com/klawr/ba/blob/master/index.html} eingesehen werden.

\section{simulation}\label{ch:simulation_js}

Die über viele Versuche wiederkehrenden Funktionalitäten sind in eine zentrale Bibliothek ausgelagert worden.
Durch die zentrale Definition wird der Aufwand zur Wartung des Codes und zur Behebung von Fehlern stark reduziert.
Dieses Prinzip ist unter dem Akronym \name{DRY}\footnote{\textit{don't repeat yourself}, zu deutsch \textit{wiederhole dich nicht}.} bekannt~\cite{AndrewHunt2021}.

Das \lstinline{simulation} Objekt beinhaltet alle Referenzen zu den HTML-Elementen, auf welche im Zuge der Versuche zugegriffen werden kann sowie den Referenzen zu den einzelnen \name{CanvasRenderingContext2D} Objekten, welche von \name{g2} genutzt werden, um auf die entsprechenden HTML-Canvas Elemente zu zeichnen.
Au{\ss}erdem werden unter Anderem globale Konstanten wie die Höhe und Breite der HTML-Canvas Elemente zentral bereitgestellt.

Auf die in \lstinline{simulation} definierten Funktionen soll im Folgenden näher eingegangen werden.
Diese bestimmen die Schnittstelle, welche die Versuchsseiten implementieren müssen.
Alle nachfolgenden Funktionen sind als Eigenschaften des \lstinline{simulation} Objektes zu verstehen.

\subsection{createElements}\label{ch:sim_createElements}

Die HTML-Elemente, welche innerhalb der Testseiten verwendet werden, sind grundsätzlich stets die Gleichen.
Aus diesem Grund kann das Befüllen der Testseiten mit den entsprechenden HTML-Elementen zentral definiert werden.
Hier werden alle Interaktionen, welche für die Versuche notwendig sind sowie vier HTML-Canvas Elemente zur einheitlichen Darstellung der Versuchsmetriken definiert.
Auf \lstinline{createElements} wird in Kapitel~\ref{ch:aufbau_der_testseiten} genauer eingegangen.

\subsection{run}\label{ch:sim_run}

Die \lstinline{run} Funktion sorgt dafür, dass alle Versuche unter vergleichbaren Bedingungen ablaufen.

\begin{lstlisting}[language=JavaScript, caption={Definition der \lstinline{simulation.run} Funktion.}, label={lst:sim_run}]
async run(step) {
    this.model?.tick(1 / 60);
    await this.g.exe(this.ctx1);
    this.time_reset = performance.now();
    step();
    this.updateTimesChart().exe(this.ctx_times);

    if (this.running) {
        this.rafId = requestAnimationFrame(() => {
            this.run(step)
        });
    }
},
\end{lstlisting}

Existiert ein \name{mec2} Modell, so wird dessen Simulation an dieser Stelle um einen Zeitschritt vorangetrieben und zur Animation verwendet.
Die Definition des \name{mec2} Modells findet in den Versuchsdateien statt.
Innerhalb der an \lstinline{run} übergebenen \lstinline{step} Funktion darf dieses Modell nicht modifiziert werden, da es als Quelle für die Bildsequenzen dient.

Anschlie{\ss}end wird der \name{CanvasRenderingContext2D} des ersten  HTML-Canavs Elements \lstinline{ctx1} gerendert\footnote{Vor der \lstinline{exe} Funktion steht hier ein \lstinline{await}. Wenn Bilder geladen werden, muss die \lstinline{exe} Funktion leicht modifiziert werden. Notwendig ist dies in der Versuchsgruppe \name{bilder2} (siehe \aka{https://klawr.github.io/ba/src/bilder/bilder2_1.html}).
Dementsprechend ist die \lstinline{run} Funktion als \lstinline{async} deklariert.}.
Auf dieses HTML-Canvas Element sollte ausschlie{\ss}lich hier zugegriffen werden, damit der Versuch nicht versehentlich Einfluss auf die Zeichnung nimmt und sich somit selbst beeinflusst.

Nachdem Rendern der Videosequenz wird die \lstinline{run} übergebene \lstinline{step} Funktion ausgeführt.
Diese Funktion implementiert den jeweiligen Versuchsalgorithmus und wird dementsprechend in den jeweiligen Kapiteln der Versuche beschrieben.

Abschlie{\ss}end wird hier noch die Dauer der \lstinline{step} Funktion vermerkt.
In \lstinline{updateTimesChart} wird der aktuelle Zeitpunkt mittels \lstinline{performance.now}\footnote{Für mehr Information zu \lstinline{performance.now} siehe \aka{https://developer.mozilla.org/en-US/docs/Web/API/Performance/now}.} ermittelt und einer Liste hinzugefügt, welche die Historie der Ausführungszeiten der einzelnen \lstinline{step} Aufrufe festhält.
Es wird au{\ss}erdem gemessen, wie viel Zeit seit dem Beginn der Aufzeichnung vergangen ist, um eine sinnvolle Darstellung der Daten im Graphen zu ermöglichen.
Dieser Graph wird anschlie{\ss}end auf einem HTML-Canvas Element durch den \lstinline{exe(ctx_times)} Aufruf gezeichnet.
\lstinline{ctx_times} ist der Kontext des vierten der von \lstinline{createElements} erstellten HTML-Canvas Elementen.

Anschlie{\ss}end wird die \lstinline{run} Funktion durch \lstinline{requestAnimationFrame} erneut aufgerufen, sofern der Versuch nicht pausiert wird\footnote{\lstinline{running} kann beispielsweise durch einen Start/Stop Knopf umgeschaltet werden.}.

\subsection{register}\label{ch:sim_register}

\lstinline{register} ist die Funktion, welche von den einzelnen Tests aufgerufen wird, um die HTML-Seite zu befüllen und die Versuchsalgorithmus zu injizieren.

In \lstinline{register} wird zunächst der Titel festgelegt, welcher dem Namen der HTML-Datei entspricht.
Dieser Titel wird in einem HTML-Anker Element platziert, damit diese in der \name{index.html} einfacher zugeordnet werden können.
Es dient au{\ss}erdem als Hyperlink, über welchen isoliert auf den Versuch zugegriffen werden kann, was sich während der Entwicklung als hilfreich herausgestellt hat.

An dieser Stelle wird au{\ss}erdem eine Funktion definiert, welche beim Laden der Seite ausgeführt wird.
Diese Funktion instanziert alle Variablen, welche für die Erstellung der Versuche notwendig sind.

Es wird zudem einmalig die \lstinline{run} Funktion aufgerufen, um das erste Bild zu rendern\footnote{Da \lstinline{simulation.running} mit den Wert \lstinline{false} initialisiert wird, wird innerhalb des \lstinline{run} Aufrufs \lstinline{requestAnimationFrame} nicht ausgeführt.}.
Die an \lstinline{register} übergebene \lstinline{step} Funktion wird an \lstinline{run} weitergereicht.
Es wird au{\ss}erdem noch der Knopf zum Starten und Pausieren der Tests definiert.
Die Betätigung dieses Knopfes schaltet mittelbar die \lstinline{simulation.running} Variable um.

Das \lstinline{simulation} Objekt enthält noch weitere Variablen und Funktionen, welche Einfluss auf den Ablauf der Tests haben.
Um die Beschreibung der \name{index.html} jedoch auf das Wesentliche zu reduzieren, sei auf den Quellcode hingewiesen, welcher unter \aka{https://github.com/klawr/ba/blob/master/src/scripts/simulation.js} eingesehen werden kann.

\section{Aufbau der Versuchsseiten}\label{ch:aufbau_der_testseiten}

Die Versuche unterscheiden sich, abgesehen von der an die \lstinline{register} Funktion übergebenen Funktion, vorallem durch die eingebundenen Dateien.
Alle Versuchsdateien binden \name{g2.full.js}, \name{mec2.min.js} und \name{simulation.js} ein.
Neben diesen existieren noch andere Hilfsbibliotheken, welche nicht von allen Versuchen genutzt werden.
Jede Versuchsdatei beinhaltet zusätzlich eine kurze Beschreibung des Algorithmus.

Wie bereits angemerkt, dient die \lstinline{createElements} Funktion der Abstraktion der Struktur aller Versuchsseiten.
Diese Funktion wurde in Kapitel~\ref{ch:sim_createElements} eingeführt und soll nun konkret beschrieben werden.

\begin{figure}
    \includegraphics[width=\textwidth]{gfx/canvasses.png}
    \caption[Bild der Canvasse]{Beispiel für die Canvasse. Hier wird der Versuch \name{momentanpol1\_3} gezeigt. Von links nach rechts: Bild, welches den Eingang darstellt. Bild, welches visualisiert was in dem Test untersucht wird. Darstellung gesammelter Daten innerhalb des Tests. Bild zur Ermittlung der Performanz des Versuchs.}\label{fig:canvasses}
\end{figure}

Die vier erstellten HTML-Canvas Elemente sollen in den Versuchen jeweils vergleichbare Rollen übernehmen.
Das Erste wird dazu verwendet den Versuchsalgorithmus mit Bildern zu versorgen.
Im zweiten HTML-Canvas Element werden die Ergebnisse des Versuchsalgorithmus visualisiert.
Die Visualierung der Daten hilft bei der Beurteilung des Versuchserfolgs und offenbart möglicherweise Erkenntnisse.
Das dritte HTML-Canvas Element soll Statistiken über die ermittelten Daten zeigen.
In den meisten Tests wird dieses Element genutzt, um die ermittelten Koordinaten in einem Graphen darzustellen.
Das Rendering des zweiten und dritten HTML-Canvas Element wird durch die \lstinline{step} Funktion übernommen.
Das Befüllen des vierten HTML-Canvas Elementes wurde bereits in Kapitel~\ref{ch:sim_run} beschrieben.

\section{Definition eines \name{mec2} Modells}

In den meisten Versuchen wird ein \name{mec2} Modell verwendet, um eine kontrollierte Videosequenz eines Mechanismus zu erstellen.
Hier wird die Beschreibung eines Pendels durch \name{mec2} gezeigt.
Eingliedrige Modelle dieser Art werden in der ersten Versuchsreihe verwendet.
Es wird durch den Drehpunkt, den Endpunkt und deren konstante Länge zueinander definiert.
Bewegt wird das Modell, sofern es keinen Antrieb hat, lediglich durch das Gewicht des Endpunktes.
Ein Beispiel für die Definition eines solchen Modells wird in Listing~\ref{lst:pendel} und in Abbildung~\ref{fig:index.html} gezeigt.

\begin{lstlisting}[language=JavaScript, caption={Definition eines Pendels in \name{mec2} inklusive Animation.}, label={lst:pendel}]
simulation.model = {
    gravity: true,
    nodes: [
        { id: 'A0', x: 150, y: 100, base: true },
        { id: 'A1', x: 230, y: 130 }
    ],
    constraints: [
        { id: 'a', p1: 'A0', p2: 'A1', len: { type: 'const' } }
    ]
};
\end{lstlisting}

Das \name{mec2} Modell wird in der entsprechenden Versuchsdatei definiert, damit dieses in den in Kapitel~\ref{ch:simulation_js} beschriebenen Funktionen korrekt referenziert werden kann.

Dieses Modell besteht aus zwei \lstinline{nodes}, welche die Endpunkte des Pendels darstellen.
Einer dieser beiden Punkte wird als \lstinline{base} deklariert, was ihn mit dem Gestell verbindet.
Aus diesem regulären JavaScript Objekt wird ein \name{mec2} Modell, indem es durch \lstinline{mec.model.extend} erweitert wird.
Dieser Vorgang wurde bereits in Kapitel~\ref{ch:sim_register} beschrieben.
Die Animation wird durch die \lstinline{simulation.run} Funktion gesteuert, welche ebenfalls in Kapitel~\ref{ch:sim_run} beschrieben wurde.