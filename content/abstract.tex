% !TEX root = ../document.tex
%
\pdfbookmark[0]{Abstract}{Abstract}
\addchap*{Zusammenfassung}
\label{sec:abstract}

In dieser Arbeit wird untersucht, welche Information anhand der Unterschiede einzelner Bilder in einer Videosequenz genutzt werden kann um einen sich darin befindlichen planaren Mechanismus zu rekonstruieren.
Die dafür durchzuführenden Versuche werden mit gängigen Webtechnologien implementiert.
Hierfür werden die Pole der ebenen Bewegung für sich bewegende Glieder ermittelt.
Außerdem werden Mittel gesucht mit denen die unterschiedlichen Glieder eines Mechanismus getrennt werden können, sodass diese isoliert untersucht werden können.
Schlussendlich werden die daraus ermittelten Pole der ebenen Bewegung genutzt um ein Modell des Mechanismus zu erstellen.

\vspace*{20mm}

{\usekomafont{chapter}Abstract}
\label{sec:abstract-diff}

This work investigates what information can be used from the differences between individual images in a video sequence to reconstruct a planar mechanism within it.
The experiments to be carried out for this purpose are implemented with common web technologies.
For this purpose, the poles of planar motion for moving links will be determined.
In addition, means are sought with which the different links of a mechanism can be separated so that they can be examined in isolation.
Finally, the poles of plane motion determined from this are used to determine a model of the mechanism.
