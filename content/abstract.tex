% !TEX root = ../document.tex
%
\pdfbookmark[0]{Abstract}{Abstract}
\addchap*{Zusammenfassung}
\label{sec:abstract}

In dieser Arbeit wird untersucht, welche Informationen anhand der Unterschiede einzelner Bilder in einer Videosequenz gewonnen werden kann, um einen sich darin befindlichen planaren Mechanismus zu rekonstruieren.
Die dafür durchgeführten Versuche sind mit gängigen Webtechnologien implementiert.
Hierfür werden die Pole der ebenen Bewegung für sich bewegende Glieder ermittelt.
Au{\ss}erdem werden Mittel gesucht mit denen die unterschiedlichen Glieder eines Mechanismus getrennt, so dass diese isoliert untersucht werden können.
Schlussendlich werden die daraus ermittelten Pole der ebenen Bewegung genutzt, um ein Modell des Mechanismus zu erstellen.

\vspace*{20mm}

{\usekomafont{chapter}Abstract}
\label{sec:abstract-diff}

This work investigates what information can be obtained from the differences between individual images in a video sequence in order to reconstruct a planar mechanism within it.
The experiments carried out for this purpose are implemented with common web technologies.
For this purpose, the poles of planar motion are determined for moving links.
In addition, means are sought with which the different links of a mechanism can be separated so that they can be examined in isolation.
Finally, the resulting poles of plane motion are used to create a model of the mechanism.
