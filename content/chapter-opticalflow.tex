% !TEX root = ../document.tex

\chapter{Ermittlung des optischen Flusses}

Die wahrgenommene Bewegung hervorgerufen durch zwei Bilder in welcher die enthaltenen Objekte sich relativ zu einander Bewegen zu scheinen wird als \name{optischer Fluss}(zu engl. \name{Optical Flow}) bezeichnet. // TODO Zitat aus Modern AI, S. 939 unten.
Inspiriert durch diesen Ansatz scheint es naheliegend ihn auf planare Mechanismen anzuwenden um so aus einer Videoaufnahme, oder mehrerer Bildaufnahmen eines Mechanismus diesen synthetisieren zu können.

% Hier ein Bild von Thomas Brox scheint angebracht. (Modern AI S. 941)

\name{Optical Flow} Methoden können genutzt werden um ein Vektorfeld mit Position, Richtung und Geschwindigkeit der sich bewegenden Pixel zu identifizieren.

Innerhalb der sich bewegenden Pixel sollen nun die Gruppen ermittelt werden, welche sich trotz absoluter Bewegung sich nicht relativ zueinander bewegen.
Ebenso sind Vektoren zu gruppieren, dessen Bewegungen sich homogen zueinander zu verhalten scheinen, es also davon ausgegangen werden kann, dass diese zu einem Glied gehören.

Bei einem stationärem Schwungpendel wäre es zu erwarten, dass die absolute Geschwindigkeit der Pixel um den Drehpunkt gering ist und linear zum Abstand des Drehpunktes zunimmt.
Ein Vergleich der Richtungsvektoren sollte allerdings mit entsprechender Skalierung gleich sein, sodass eine Gruppierung hier eindeutig ist.

Diese Gruppen können dann weiter daraufhin untersucht werden ob sie sich innerhalb bestimmter Rahmenbedingungen zueinander bewegen.
Das bedeutet, ob sie umeinander rotieren, oder gegebenenfalls die Gruppe selber sich in ihrer Länge ändert, was einer translatorischen Bewegung zugeordnet werden kann.

% NOTES:
% Performance increase through "superpixels", superpixels klingt nach etwas, was ich auch haben will. Siehe: Contour Detection and Hierarchical Image Segmentation