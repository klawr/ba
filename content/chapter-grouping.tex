\chapter{Gruppierung von Datenpunkten zu Gliedern}

\cleanchapterquote{Kinematische Kette ist die Aneinanderreihung wenigstens dreier durch Elementenpaare beweglich miteinander verbundener Glieder}{VDI2127}{1.5}

Im vorangegangen Abschnitt wurde die Erkennung von einzelnen Gliedern durch ihre Drehung behandelt.
Alle Methoden gingen jedoch davon aus, dass sich alle Datenpunkte einem Glied zuordnen lassen.
Es ist entsprechend notwendig bevor man die bereits besprochenen Methoden vornimmt eine Zuordnung der Punkte in entsprechende Gruppen zu unternehmen, welche dann an die bekannten Methoden weiter gegeben werden können.

Für eine solche Gruppierung können verschiedene Ansätze herangezogen werden.

\section{Zuordnung von Linien an Punkte}

Der erste Ansatz welcher untersucht wird geht von der Annahme aus, dass sich alle Punkte in etwa um sich auf Linien zurückzuführende Glieder herum befinden.
Diese Annahme ist ähnlich wie bei den vorangegangen Methoden, dass die Glieder immer länger sind als sie breit sind.

Hierfür wird ein Punkt aus den Datenpunkten genommen und dann eine Linie definiert welche am ehesten dem Glied entspricht welche diesem Punkt zugeordnet werden kann.
Dann werden alle Punkte genommen dessen Abstand zu dieser Linie kleiner ist als ein Schwellenwert, welcher vorher definiert werden muss.

Alle Punkte dessen Abstand unter diesem Schwellenwert sind werden dann der ersten Gruppen zugeordnet und der Prozess wird mit allen ungruppierten Punkten wiederholt.

