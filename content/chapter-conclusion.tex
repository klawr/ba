% !TEX root = ../document.tex

\chapter{Zusammenfassung und Ausblick} \label{ch:zusammenfassung_ausblick}

\cleanchapterquote{Wenn man die Daten lange genug foltert, werden sie alles gestehen. }{Ronald Harry Coase}{Britischer Ökonom}

In dieser Arbeit wurde untersucht, welche Informationen einer Bildsequenz von einem sich bewegendem planaren Mechanismus gewonnen werden kann.
Es hat sich gezeigt, dass allein durch die Punktwolke der sich verändernden Pixel einiges ermittelt werden kann.
Nachdem für die Versuche über den kleinsten umfassenden Kreis zwar viele nützliche Methoden entwickelt wurden, so hat dieser in einer abschlie{\ss}enden Lösung keinen Platz gefunden.
Die Ermittlung der orthogonalen Regressionsgeraden hingegen hat sich hier als ein notwendiges Hilfsmittel herausgestellt, welches die Gliedebenen selber definiert und für die Ermittlung der Winkelgeschwindigkeiten genutzt wird.
Au{\ss}erdem wurden Methoden entwickelt, um die zunächst unzusammenhängenden Punkte über die zeitlichen Iterationen miteinander zu verbinden, um so auch darüber Informationen gewinnen zu können.

Die Nutzung der äu{\ss}ersten Punkte der Punktwolke wurde in späteren Versuchen durch den \name{Shi-Tomasi} Algorithmus ersetzt.
Diese Punkte werden dann durch eine Implementation des \name{Lucas-Kanade} Algorithmus verfolgt, so dass aus diesen die Absolutgeschwindigkeit von auf den Gliedern ermittelten Punkten gemessen wird.
Die Anwendung dieser Algorithmen hat hier den Fehler bereits ma{\ss}geblich reduziert, so dass bei einzelnen Gliedern die Momentanpole mit einer hinreichenden Genauigkeit ermittelt werden konnten.

Die Absolutgeschwindigkeit und die Winkelgeschwindigkeit werden genutzt, um den Momentanpol eines sich in der Ebene bewegenden Gliedes zu ermitteln.
Nach der Ermittlung aller Momentanpole lassen sich dann daraus jene filtern, welche als Absolutpole interpretiert werden können.
Daraufhin werden kinematische Gesetzmä{\ss}igkeiten angewandt, um nicht als Gelenk nutzbare Relativpole zu filtern, um den zugrundeliegenden Mechanismus zu rekonstruieren.

Die vorgeschlagenen Methoden beinhalten alle Messungenauigkeiten, welche es zunächst nicht zulassen kompliziertere Mechanismen zu rekonstruieren.
Allerdings wurden Methoden aus dem Themengebiet der künstlichen Intelligenz vorgestellt, welche diese Fehler durch eine bessere Segmentierung der Glieder reduzieren können.
Diese wurden jedoch auch zugunsten des Rahmens dieser Arbeit nicht weiter unteruscht.

Für Versuche mit reellen Mechanismen wurden ebenfalls Versuche durchgeführt.
Mit der Nutzung von einer Kamera mit Stativ unter Tageslicht waren die Störungen jedoch zu gro{\ss}, als dass die \lstinline{compareImages} Funktion sinnvolle Information ermitteln konnte\footnote{Der entsprechende Versuch ist unter \aka{https://klawr.github.io/ba/src/bilder/bilder1_3.html} zu finden.}.
Hier ist es denkbar die Methoden zur Ermittlung von Änderung weniger sensibel zu machen um so die gesuchten Änderungen hervorzuheben.

Es wurden au{\ss}erdem Versuche mit dem in \name{mecEdit}~\cite{Uhlig2019} enthaltenen \name{Pumpjack} durchgeführt, welche in der Versuchsgruppe \name{bilder} enthalten sind.
Die Ergebnisse sind in Abbildung~\ref{fig:bilder2} zu sehen.

\begin{figure}
    \centering
    \includegraphics[width=\textwidth]{gfx/bilder2_edited.png}
    \caption[Versuche \name{bilder2\_3.html}, \name{bilder2\_4.html} und \name{bilder2\_6.html}]{Versuche \name{bilder2\_3.html}, \name{bilder2\_4.html} und \name{bilder2\_6.html}. Im ersten Canvas wird der Pumpjack gezeigt~\cite{Uhlig2021}. Im zweiten Bild wird der \name{k-Means} Algorithmus aus Kapitel~\ref{ch:kMeans} angewandt. Im dritten Bild wird der \name{Lucas-Kanade} Algorithmus aus Kapitel~\ref{ch:lucas_kanade} genutzt. Im letzten Bild wird der \name{Dijkstra} Algorithmus aus Kapitel~\ref{ch:dijkstra} gezeigt.}
    \label{fig:bilder2}
\end{figure}

Schlussendlich ist zu sagen, dass eine komplette Rekonstruktion von Mechanismen durchaus denkbar ist.
Die dafür notwendigen Technologien existieren, es bedarf an dieser Stelle nur einer sehr viel feinere Einstellung der Parameter oder entsprechend komplexeren Verfahren.
