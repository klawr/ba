% !TEX root = ../document.tex
%

\chapter{Einleitung}
\label{ch:einleitung}

In dieser Arbeit wird die automatisierte Erkennung von planaren Mechanismen durch wahrgenommene Bewegung in Bildsequenzen versucht.

\begin{figure}
\centering
    \includegraphics[width=\textwidth]{gfx/werkzeugkoffer_impl.png}
    \caption[Versuch \name{bilder1\_1.html}]{Versuch \name{bilder1\_1.html}. Im linken Bild wird der Mechanismus mit darüberliegendem \name{mec2} Modell gezeigt. Das mittlere Bild zeigt den ursprünglichen Mechanismus. Im rechten Bild ist zur verbesserten Ansicht nur das \name{mec2} Modell zu sehen. Ein Mensch ist durch Erfahrung und Intuition in der Lage diesen Mechanismus direkt zu modellieren.}
    \label{fig:werkzeugkoffer_impl}
\end{figure}

Die Lösung des Problems ist interessant, da diese Aufgabe typischerweise von einem Menschen auch ohne Hintergrund in den Ingenieurswissenschaften problemslos gelöst werden kann.
Selbst ohne Bewegung würde ein Mensch vermutlich mit geringen Abweichungen die beweglichen Gelenkpunkte und dessen vermeintliche Bewegung bestimmen können.
Hierfür wird auf Erfahrung und Intuition zurückgegriffen, über die ein Programm typischerweise nicht verfügt.
Neuronale Netzwerke könnten sicherlich trainiert werden, um die Gelenkpunkte eines Mechanismus anhand einer Momentaufnahme zu bestimmen.
In dieser Arbeit soll jedoch untersucht werden, welche Informationen über planare Mechanismen innerhalb von Videosequenzen auf analytischem Wege gewonnen werden können.

Für die Bestimmung der Posen bekannter kinematischer Ketten existieren Lösungen, welche bereits in der Praxis angewandt werden.
Diese Ansätze werden unter dem Begriff der \name{Pose-Estimation} zusammengefasst.
Besonders prominent sind hierrunter Ansätze, welche die Pose eines auf Menschen definierten Modells ermitteln und visualisieren~\cite{Papandreou2018, Google2021, Google2021a}.
Im Unterschied zur Problemstellung dieser Arbeit, in der die kinematischen Ketten ermittelt werden sollen, gehen solche Methoden von bereits bekannten kinematischen Ketten aus.

\begin{figure}
    \centering
    \begin{subfigure}[t]{0.3\textwidth}
        \includegraphics[width=\textwidth]{gfx/posenet.png}
    \end{subfigure}
    \begin{subfigure}[t]{0.3\textwidth}
        \includegraphics[width=\textwidth]{gfx/movenet.png}
    \end{subfigure}
    \caption[Posenbestimmung durch maschinelles Lernen]{Ermittlung der Pose einer kinematischen Kette auf Modellbasis von Menschen durch \name{PersonLab}~\cite{Google2021} (links) und \name{MoveNet}~\cite{Google2021a} (rechts).}\label{fig:movenet}
    \label{fig:pose_estimation}
\end{figure}

Die Erkennung der Bewegung soll durch den Vergleich von Bildern in Videosequenzen erfolgen.
Ziel ist das Untersuchen von Algorithmen, welche die Bewegung einzelner Glieder ermitteln.
Die Ergebnisse sollen dann genutzt werden, um die Pole der ebenen Bewegung für die entsprechenden Glieder zu finden.
Da ein Mechanismus aus mehreren Gliedern besteht, müssen Methoden entwickelt werden, um diese Glieder zu isolieren.
Anhand der Gliedebenen und derer Momentanpole werden die übrigen Relativpole ermittelt.
Unter den Relativpolen können dann die Gelenkpunkte gesucht werden.

Es werden folgende Annahmen getroffen, welche teilweise durch die Nutzung der Physik-Engine \name{mec2} vorgegeben sind.
So sind die Glieder des Mechanismus inkompressibel und unverbieglich.
Entsprechend kann auch Knickung nicht auftreten.
Das Gestell ist unbeweglich und auch keines der Gelenke weist Spiel auf.
Des Weiteren wird angenommen, dass alle translativen Elemente als Gestellglieder auftreten.

Die Mechanismen werden mit Hilfe von \name{mec2} simuliert, deren Animation daraufhin als Videosequenz untersucht werden.
Die Verfahren sind alle in den Standardwebtechnologien \name{HTML} und \name{JavaScript} implementiert.
Die entsprechenden Versuche können unter \aka{https://klawr.github.io/ba} gefunden werden.
Alle Versuche wurden mit dem Webbrowser \name{Mozilla Firefox 91.0.1} auf \name{Windows 10} durchgeführt.
