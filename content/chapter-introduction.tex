% !TEX root = ../document.tex
%

\chapter{Einleitung}
\label{ch:einleitung}

In dieser Arbeit wird die automatisierte Erkennung von planaren Mechanismen durch wahrgenommene Bewegung in Bildsequenzen versucht.

\begin{figure}
\centering
    \includegraphics[width=\textwidth]{gfx/werkzeugkoffer_impl.png}
    \caption[Versuch \name{bonus1\_1.html}]{Versuch \name{bonus1\_1.html}. Im linken Bild wird der Mechanismus mit darüberliegendem \name{mec2} Modell gezeigt. Das mittlere Bild zeigt den ursprünglichen Mechanismus. Im rechten ist zur verbesserten Ansicht nurnoch das \name{mec2} Modell zu sehen. Ein Menschen ist durch Erfahrung und Intuition in der Lage diesen Mechanismus auch ohne Bewegung zu modellieren.}\label{fig:werkzeugkoffer_impl}
\end{figure}

Die Lösung des Problems ist interessant, da diese Aufgabe von einem Menschen, typischerweise auch ohne Hintergrund in den Ingenieurswissenschaften oder der Mechanismentechnik, problemslos gelöst werden kann.
Selbst ohne Bewegung würde vermutlich mit einer niedrigen Toleranz bemessen ein Mensch die beweglichen Gelenkpunkte und dessen vermeintliche Bewegung bestimmen können.
Hierfür können sie auf Erfahrung und Intuition zurückgreifen, über die ein Programm typischerweise nicht verfügt.
Moderne Techniken innerhalb der künstlichen Intelligenz könnten sicherlich trainiert werden um die Gelenkpunkte eines Mechanismus anhand einer Momentaufnahme zu bestimmen, jedoch soll hier zunächst versucht werden herauszufinden welche Regelmä{\ss}igkeiten sich innerhalb der Problemstellung finden können ohne ein neurales Netzwerk zu erstellen.

Für die Bestimmung von Posen gibt es bereits Lösungen welche schon Anwendung finden.
Diese versammeln sich unter dem Begriff der \name{Pose-Estimation}.
Besonders prominent sind hierrunter Ansätze welche die Pose von Menschen ermitteln und entsprechend anzeigen~\cite{Papandreou2018, Google2021, Google2021a}.
Der Unterschied zu diesen Ansätzen ist, dass diese die bereits bekannten mechanischen Modelle auf ein Bild im realen Leben applizieren.
Dies steht dem in dieser Arbeit untersuchten Ansatz gegenüber bei dem das mechanische Modell eben unbekannt ist.

\begin{figure}
    \centering
    \begin{subfigure}[t]{0.3\textwidth}
        \includegraphics[width=\textwidth]{gfx/posenet.png}
        \caption{Ermittlung der Pose eines Menschen durch \name{PersonLab}~\cite{Google2021}.}\label{fig:posenet}
    \end{subfigure}
    \begin{subfigure}[t]{0.3\textwidth}
        \includegraphics[width=\textwidth]{gfx/movenet.png}
        \caption{Ermittlung der Pose eines Menschen durch \name{MoveNet}~\cite{Google2021a}.}\label{fig:movenet}
    \end{subfigure}
    \caption[Posenbestimmung durch maschinelles Lernen]{}
    \label{fig:pose_estimation}
\end{figure}

Die Erkennung der Bewegung soll durch den Vergleich von Bildern in Videosequenzen erfolgen; die Modellierung geschieht durch die Bestimmung der Glieder und Gelenke des entsprechenden Mechanismus.

Ziel ist es Möglichkeiten zu untersuchen welche es erlauben die Bewegung einzelner Glieder zu ermitteln.
Die Ergebnisse sollen dann genutzt werden um die Pole der ebenen Bewegung für die entsprechenden Glieder zu finden.
Da der Mechanismus aus mehreren Gliedern besteht müssen dann Wege ermittelt werden, wie diese Glieder isoliert werden können, damit entsprechend diese auf ihre Pole untersucht werden können.
Anhand der Gliedern und dessen ermittelten Momentanpole sollten die übrigen Relativpole ermittelt werden können.
Diese müssen dann nach jenen gefiltert werden, welche als Gelenk dienen.

Innerhalb der Lösung werden zunächst mehrere Annahmen getroffen, welche durch die Nutzung der Physik-Engine \name{mec2} vorgegeben werden.
So sind die Glieder des Mechanismus inkompressibel und unverbieglich.
Entsprechend kann auch eine Knickung nicht auftreten.
Das Gestell ist unbeweglich und auch keines der Gelenke weist Spiel auf.

Hierfür wird zunächst ein Konstrukt vorgestellt, welches es erlaubt viele Versuche mit geringem Mehraufwand hinzuzufügen.
Hierfür werden mit Hilfe von \name{mec2} Mechanismen erstellt, welche daraufhin von entsprechenden Algorithmen rekonstruiert werden sollen.
Die Verfahren sind alle in den Standard-Webtechnologien \name{HTML} und \name{JavaScript} geschrieben.
Die entsprechenden Versuche können unter \aka{https:klawr.github.io/ba} gefunden werden.
