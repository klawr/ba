% !TEX root = ../document.tex
%

\chapter{Einleitung}
\label{ch:einleitung}

In dieser Arbeit wird die automatisierte Erkennung von planaren Mechanismen durch wahrgenommene Bewegung in Bildsequenzen versucht.

% TODO hier ein Bild von einem Mechanismus einfügen, der zeigt das der Leser in der Tat ziemlich genau bestimmen kann wie dieser aussehen würde.
Die Lösung des Problems ist interessant, da diese Aufgabe von einem Menschen, typischerweise auch ohne Hintergrund in den Ingenieurswissenschaften oder der Mechanismentechnik, problemslos gelöst werden kann.
Selbst ohne Bewegung würde vermutlich mit einer niedrigen Toleranz bemessen ein Mensch die beweglichen Gelenkpunkte bestimmen und vorhersagen können wie diese sich bewegen würden, würden diese angetrieben.
Um das tun, greifen Menschen auf einen Fundus an Erfahrung und daraus folgender Intuition zurück, über die ein Algorithmus nicht verfügt.

% TODO wurde das denn schonmal versucht?
Moderne Techniken innerhalb der künstlichen Intelligenz könnten sicherlich trainiert werden um die Gelenkpunkte eines Mechanismus anhand einer Momentaufnahme zu bestimmen, auf diese wird in dieser Arbeit jedoch nicht eingengangen.

Die Erkennung erfolgt durch den Vergleich von Bildern in Videosequenzen; die Modellierung geschieht durch die Bestimmung der Glieder und Gelenke des entsprechenden Mechanismus.

Ziel ist es dann ein Programm zu schreiben welches, bei entsprechendem Videoeingang ein Modell erstellt, dessen Bewegung innerhalb der von \name{mec2} gegebenen Physik-Engine äquivalent zu dem beobachtetem Mechanismus ist.
Die \name{mec2} Physik-Engine wird gewählt, weil sie eine direkte Definition der Mechanismen innerhalb der von Menschen und auch dem Computer lesbaren JSON-Formats sofortige Resultate liefern kann.
Das ermöglicht es die Algorithmen die im Rahmen dieser Arbeit entworfen werden direkt zu testen, da bereits der Videoeingang modelliert werden kann und die Lösung bekannt ist.

Innerhalb der Lösung werden zunächst mehrere Annahmen getroffen, welche durch die Nutzung der Physik-Engine \name{mec2} vorgegeben werden.
So sind die Glieder des Mechanismus inkompressibel und haben auch keine Biegung.
Entsprechend kann auch eine Knickung nicht auftreten.
Die festen Gelenkpunkte sind unbeweglich und auch keines der Gelenke hat ein Spiel.

Der erste Ansatz der untersucht wird ist der Abgleich zweier aufeinanderfolgenden Bilder, worin die Unterschiede ermittelt werden.
Hier wird untersucht wie sich Bewegung in einer Videosequenz bemerkbar macht und sie entsprechend genutzt werden kann.
Hierfür werden mit Hilfe von \name{mec2} Mechanismen erstellt, welche daraufhin von entsprechenden Algorithmen rekonstruiert werden sollen.
Begonnen wird mit der Ermittlung eines einfach Pendels.
Es wird sich zeigen das ein Schwerpunkt der Aufgabe darin liegen wird, den Drehpunkt einzelner Glieder zu ermitteln, nachdem die zu ermittelnden Glieder differenziert werden.

Daraufhin sollen andere Ansätze untersucht werden, welche Bewegung innerhalb aufeinanderfolgenden Bildern zu ermitteln versuchen.
Die Untersuchung der Bewegung soll dann Aufschlüsse darüber geben, ob diese einem Mechanismus zugeordnet werden können.
Einer dieser Ansätze ist der Lucas-Kanade Algorithmus, welcher implementiert und angepasst wird. % TODO citation
Es wird sich zeigen, dass dies einen Schwerpunkt dieser Arbeit darstellt.
Dann von der gewonnenen Information den ursprünglichen Mechanismus zu rekonstruieren ist dann die Hauptaufgabe.

Die Verfahren sind alle in den Standard-Webtechnologien \name{HTML} und \name{JavaScript} geschrieben.
% TODO prepare this website...
Die Resultate der einzelnen Schritte der jeweiligen Abschnitte können über \aka{https://klawr.github.io/ba/index.html} nachvollzogen werden.
