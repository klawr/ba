\chapter{Ermittlung des Momentanpols}

\cleanchapterquote{Die allgemeine ebene Starrkörperbewegung kann augenblicklich als reine Drehung um einen ausgezeichneten Punkt -- den Momentanpol oder Geschwindigkeitspol -- aufgefasst werden}{Stefan Gössner}{Mechanismentechnik}

Im vorangegangenem Kapitel wurde die Ermittlung eines Drehpunktes von sich einem rotierendem Glied beschrieben.
Diese Bewegung stellt einen häufigen Fall in Mechanismen dar, jedoch ist der feste Drehpunkt ein Spezialfall für die Bestimmung des Momentanpols eines Gliedes.

Der Momentanpol ist definiert als der Punkt, welcher sich bei Bewegung eines Elementes nicht bewegt.
Das kann auch interpretiert werden als der Punkt um den sich ein Element in dieser Momentbetrachtung dreht.
Dieser kann im Bezug zum Koordinatensystem konstant sein, was einen Drehpunkt definiert.
Ein translatives Elemtent würde sich um keinen Punkt drehen, was entsprechend einen Momentanpol entspricht welcher eine unendliche Distanz zum betrachteten Element hat.

Der Momentanpol eines Gliedes befindet sich in einem Mechanismus stets im Schnittpunkt der anliegenden Glieder.
Diese Gesetzmäßigkeit wird \name{Satz der konjugierten Krümmungsmittelpunkte} % TODO \cite{Gössner2017}
Für ein Viergelenk würde dies Bedeuten, dass der Momentanpol der Koppel sich stets im Schnittpunkt der (virtuell verlängerten) Kurbel und Schwinge befindet.

Bei einer Parallelkurbel bei welcher sich die Koppel nicht bewegt, würde der Momentanpol stets im Unendlichen liegen.
Entsprechend findet bei dieser auch keine Drehung statt.

Eine weitere Möglichkeit der Ermittlung des Momentanpols ist die Ermittlung der Bewegung durch den \name{ersten Satz von Euler}. % TODO \cite{Gössner2017}
Die Bewegung eines Körpers berechnet sich nach diesem durch

\begin{equation}
    \vec{v}_A = \vec{v}_P + \omega \tilde{\vec{r}}_{PA}
    \label{eq:satz_von_euler}
\end{equation} 

Hier ist $\vec{v}_A$ die momentane Geschwindigkeit eines Körperpunktes $A$.
$\vec{v}_P$ ist die Geschwindigkeit des Körperpunktes $P$.
$\omega\tilde{\vec{r}}_{PA}$ entspricht der Geschwindigkeit des Punktes $P$ nach $A$ durch % TODO \cite{Gössner2017, p.75}

\begin{equation}
    \vec{v}_{AB} = \dot{\vec{r}}_{AB} = \omega \tilde{\vec{r}}_{AB}.
    \label{eq:absolut_zu_winkel}
\end{equation}

Wird für Gleichung~\ref{eq:satz_von_euler} der Punkt $P$ nun als der Momentanpol festgelegt, kann die Geschwindigkeit $\vec{P} = 0$ gesetzt werden,
sodass sich Gleichung~\ref{eq:satz_von_euler} zu

\begin{equation}
    \vec{v}_A = \omega \tilde{\vec{r}}_{PA}
    \label{eq:satz_von_euler_momentanpol}
\end{equation}

vereinfacht. 
Diese kann dann nach $r_{AP}$ umgestellt werden,

\begin{equation}
    \vec{r}_{AP} = \frac{\tilde{\vec{v}_A}}{\omega}
\end{equation}

sodass der Momentanpol $P$ über die Beziehung $\vec{r}_P = \vec{r}_A + \vec{r}_{AP}$ ermittelt werden kann.

Diese Formel soll nun angewendet werden um den Momentanpol von Elementen zu bestimmen dessen Bewegung in Videos nachvollzogen wird.

\section{Das drehende Rad}

Als erster Test soll der Momentanpol eines sich drehenden Rades bestimmt werden.
Von diesem Rad soll eine Speiche sichtbar sein, welche es erlaubt die Drehung zu erkennen.
Bekannt ist, dass die Drehung eines Rades im Wälzpunkt liegt, damit dieses Schlupffrei rollen kann.
Es wird also erwartet, dass die über die Zeit ermittelten Momentanpole eine Linie bilden, welche sich auf genau der Höhe befindet, auf welcher das Rad rollt.

Für die Animation des Rades wurde eine \name{g2} Animation erstellt.



Wenn diese Drehung korrekt ermittelt wird, sollte nach Gleichung~\ref{eq:satz_von_euler_momentanpol} der Momentanpol ermittelt werden können.

