% !TEX root = ../document.tex

\chapter{Ermittlung des Momentanpols}\label{ch:ermittlung_momentanpol}

\cleanchapterquote{Die allgemeine ebene Starrkörperbewegung kann augenblicklich als reine Drehung um einen ausgezeichneten Punkt -- den Momentanpol oder Geschwindigkeitspol -- aufgefasst werden}{Stefan Gössner}{Mechanismentechnik}

In Kapitel~\ref{ch:infoerhalt_aus_abgleich_bilder} wurde die Ermittlung eines Drehpunktes von sich einem rotierendem Glied beschrieben.
Diese Bewegung stellt einen häufigen Fall in Mechanismen dar, jedoch ist der feste Drehpunkt ein Spezialfall für die Bestimmung des Momentanpols eines Gliedes.

Der Momentanpol ist definiert als der Punkt, welcher sich bei Bewegung eines Elementes nicht bewegt.
Das kann auch interpretiert werden als der Punkt um den sich ein Element in dieser Momentbetrachtung dreht.

\begin{wrapfigure}{r}{0.5\textwidth}
    \centering
    \includegraphics[width=0.4\textwidth]{gfx/mechanismentechnik_vektorfeld_momentanpol.png}
    \caption{Diese Betrachtung eines starren Körpers in der Ebene soll nahelegen, dass sich die Bewegung auch als Vektorfeld interpretieren lässt~\cite{Goessner2017}.}\label{fig:mechanismentechnik_vektorfeld_momentanpol}
\end{wrapfigure}

Es ist zu beachten, dass der Momentanpol ein virtueller Punkt und damit nicht an den untersuchten Körper gebunden ist.
Tatsächlich kann entsprechend die Bewegung des ebenen starren Gliedes als ein Vektorfeld extrapoliert werden, was ebenfalls den entsprechenden Momentanpol als eine Koordinate ohne Bewegung innerhalb dieses vorhersagen würde.
Wären die Methoden der Berechnung des optischen Flusses also mit nur sehr kleinem Fehler behaftet, würde sich so möglicherweise auch direkt der Momentanpol ermitteln lassen.

Der Momentanpol kann im Bezug zum Koordinatensystem konstant sein, was einen Drehpunkt definiert, welcher dann auch als Absolutpol bezeichnet wird.
Ein translatives Elemtent, dessen Bewegung durch die abwesenheit einer Rotation definiert ist dreht sich in dem Sinne um einen Momentanpol welcher sich orthogonal zur Bewegungsrichtung in unendlicher Distanz befinidet.
Bei einer Parallelkurbel bei welcher sich die Koppel nicht bewegt, würde der Momentanpol stets im Unendlichen liegen.
Entsprechend findet bei dieser auch keine Drehung statt.

Eine Möglichkeit der Ermittlung des Momentanpols ist die Ermittlung der Bewegung durch den \name{ersten Satz von Euler}\cite{Gössner2017}.
Die Bewegung eines Körpers berechnet sich nach diesem durch

\begin{equation}
    \vec{v}_A = \vec{v}_P + \omega \tilde{\vec{r}}_{PA}
    \label{eq:satz_von_euler}
\end{equation} 

Hier ist $\vec{v}_A$ die momentane Geschwindigkeit eines Körperpunktes $A$.
$\vec{v}_P$ ist die Geschwindigkeit des Körperpunktes $P$.
$\omega\tilde{\vec{r}}_{PA}$ entspricht der Geschwindigkeit des Punktes $P$ nach $A$ durch

\begin{equation}
    \vec{v}_{AB} = \dot{\vec{r}}_{AB} = \omega \tilde{\vec{r}}_{AB}.
    \label{eq:absolut_zu_winkel}
\end{equation}

Wird für Gleichung~\ref{eq:satz_von_euler} der Punkt $P$ nun als der Momentanpol festgelegt, kann die Geschwindigkeit $\vec{P} = 0$ gesetzt werden,
sodass sich Gleichung~\ref{eq:satz_von_euler} zu

\begin{equation}
    \vec{v}_A = \omega \tilde{\vec{r}}_{PA}
    \label{eq:satz_von_euler_momentanpol}
\end{equation}

vereinfacht. 
Diese kann dann nach $r_{AP}$ umgestellt werden,

\begin{equation}
    \vec{r}_{AP} = \frac{\tilde{\vec{v}}_A}{\omega}
    \label{eq:euler_rAP}
\end{equation}

sodass der Momentanpol $P$ über die Beziehung $\vec{r}_P = \vec{r}_A + \vec{r}_{AP}$ ermittelt werden kann.

Diese Formel soll nun angewendet werden um den Momentanpol von Elementen zu bestimmen dessen Bewegung in Videos nachvollzogen wird.

% TODO implementation

\section{Das drehende Rad}\label{ch:drehendesRad}

Als erster Test soll der Momentanpol eines sich drehenden Rades bestimmt werden.
Von diesem Rad soll eine Speiche sichtbar sein, welche es erlaubt die Drehung zu erkennen, da der Momentanpol für die Bewegung jedes starren Körpers im Raum definiert ist und nicht auf Glieder von Mechanismen beschränkt ist.
Interessant ist die Bestimmung des Momentanpols eines Rades, weil dieses als erster der bisher gemachteten Tests keinen Absolutpol darstellt.
Bekannt ist, dass die Drehung eines Rades im Wälzpunkt liegt.
Andererseits würde dieses nicht schlupffrei rollen können\footnote{Bei einem rutschendem Rad würde sich der Drehpunkt entsprechend im Mittelpunkt des Rades befinden.}.
Es wird also erwartet, dass die über die Zeit ermittelten Momentanpole eine Linie bilden, welche sich auf genau der Höhe befindet, auf welcher das Rad rollt.

Für die Animation des Rades wurde eine \name{g2} Animation erstellt,
da sich dieses Modell so leichter darstellen lässt als mit einem \name{mec2} Modell.
Das Rad wird durch eine Linie und einen Kreis definiert.
Das Rollen wird durch den \lstinline{g2.use} Befehl simuliert, indem die Rotation durch eine Laufvariable bestimmt wird und die Position in X-Richtung durch $x = x_0 + r * i * \pi$, wobei $x_0$ entsprechend der Startwert ist, $r$ der Radius des Kreises und $i$ die Laufvariable, welche hier den Radianten bestimmt.
Die Animation wird dann an \lstinline{globalTestVariables.g} übergeben.
Da kein \name{mec2} Modell definiert ist werden dessen Befehle, wie in Kapitel~\ref{ch:simulation_js} beschrieben, ignoriert.

Zur Ermittlung des Momentanpols ist es nun notwendig die absolute Geschwindigkeit eines Punktes auf dem Glied, sowie die Rotationsgeschwindigkeit des Gliedes selbst zu ermitteln.
Die absolute Geschwindigkeit soll zunächst ermittelt werden, indem wie in Kapitel~\ref{ch:vergleich_verlauf_randpunkte} beschrieben die Bewegung der äußersten Randpunkte gemessen wird.
Unter der Vorrausetzung, dass diese in etwa den selben Punkt auf dem Glied bezeichnen, sollten diese eine ausreichende Approximation darstellen.
Die \lstinline{Group} Klasse speichert hierbei den Pfad der gemessenen Punkte und ordnet diese entsprechend zu.
Durch den Gradienten zweier Punkte in dieser Liste kann so eine Geschwindigkeit berechnet werden.
Die Rotationsgeschwindigkeit wird ähnlich gemessen.
Hierfür hat die \lstinline{Group} Klasse die Eigenschaft \lstinline{lines} bekommen, welche in etwa die selbe Aufgabe wie die \lstinline{pts} Eigenschaft erfüllen soll.
Wird der Gradient über die selbe Anzahl an Iterationen für die Winkelgeschwindigkeit wie für die absolute Geschwindigkeit berechnet, so kann Gleichung~\ref{eq:euler_rAP} genutzt werden um die Position des Momentanpols zu ermitteln.

Dieser stellt hier wieder eine Schätzung dar, da die Eingangsparameter mit einer Ungenauigkeit behaftet sind.
Des Weiteren ist anzumerken, dass die Anzahl der Iterationen zwischen den beiden zu vergleichenden Bildern im vorhinein festgelegt werden muss.
Ist diese Zahl zu klein, oder zu groß, so folgt daraus eine größere Standardabweichung.
Diese optimale Anzahl ist jedoch je nach Geschwindigkeit des sich bewegenden Gliedes unterschiedlich, sodass zur Ermittlung dieses weitere Untersuchung notwendig sind.
Für den Versuch des Rades wird hier lediglich die Y-Koordinate betrachtet.
Tatsächlich befindet sich der Erwartungswert des Momentanpols, unter Berücksichtigung der Standardabweichung, in etwa im Wälzpunkt des Rades.

\begin{figure}
    \centering
    \begin{subfigure}[t]{0.45\textwidth}
        \includegraphics[width=\textwidth]{gfx/drehendes_rad_1.png}
        \caption{Versuch: \name{momentanpol1\_1.html}.}\label{fig:drehendes_rad_1}
    \end{subfigure}
    \begin{subfigure}[t]{0.45\textwidth}
        \includegraphics[width=\textwidth]{gfx/drehendes_rad_4.png}
        \caption{Versuch: \name{momentanpol1\_4.html}.}\label{fig:drehendes_rad_4}
    \end{subfigure}
    \caption{Die orangenen Punkte sind hier die berechneten Momentanpole der entsprechenden Iteration. Der Verlauf ist von links nach rechts, entsprechend der Bewegung des Rades. Die Graphen zeigen jeweils die Verteilung der Momentanpole gemäß ihrer Y-Koordinate.}
    \label{fig:drehendes_rad_1_4}
\end{figure}

Weitere Versuche wurden dann unternommen indem nicht mehr die am weitesten voneinander entfernten Punkte, sondern die in Kapitel~\ref{ch:lucas_kanade} verwendete Methode genutzt wird.
Hierfür wird zunächst ein Punkt über den Shi-Tomasi Algorithmus gewählt.
Dieser nutzt die beste sichtbare Kante, welche in diesem Fall den von \name{g2} definierten \lstinline{nod} verursacht wird.
Der Verlauf der nun verfolgten Punkte verspricht weniger Streuung der Koordinaten und entsprechend eine höhere Genauigkeit bei der Berechnung der Geschwindigkeit.

Des Weiteren soll nicht mehr die Gerade der am weitesten auseinander liegenden Punkte genutzt werden, sondern die orthogonale Regressionsgerade, welche in Kapitel~\ref{ch:orthogonale_regression} beschrieben wurde.
Diese verspricht ebenfalls weniger Streuung bei der Ermittlung des vom Glied eingelegten Winkels.

Die Berechnung des Momentanpols ist ansonsnten äquivalent.
In Abbildung~\ref{fig:drehendes_rad_1_4} ist zu sehen, dass die Berechnung in der Tat eine erheblich geringere Streuung aufweisen.
Die gezeigten Ansätze überspringen jeweils 20 Iterationen.
Dieser Wert ist experimentell ermittelt und sollte der Geschwindigkeit der Drehung angepasst werden.
Die Standardabweichung hat sich durch den neuen Ansatz von $4.8$ auf $1.9$ reduziert.

\section{Gestellglieder}

\subsection{Drehendes Gestellglied}

Diese herangehensweise soll nun für die Ermittlung des Drehpunktes eines wie in Kapitel~\ref{ch:infoerhalt_aus_abgleich_bilder} betrachteten Pendels genutzt werden.
Der Drehpol eines Gliedes welches mit dem Gestell verbunden ist sollte sich im Momentanpol befinden.
Für diesen Versuch wurde entsprechend das Pendel aus Kapitel~\ref{ch:infoerhalt_aus_abgleich_bilder} wiederverwendet.

\begin{figure}
    \centering
    \begin{subfigure}[t]{0.45\textwidth}
        \includegraphics[width=\textwidth]{gfx/drehendes_pendel_1.png}
        \caption{Versuch: \name{momentanpol2\_1.html}.}\label{fig:drehendes_pendel_1}
    \end{subfigure}
    \begin{subfigure}[t]{0.45\textwidth}
        \includegraphics[width=\textwidth]{gfx/drehendes_pendel_4.png}
        \caption{Versuch: \name{momentanpol2\_4.html}.}\label{fig:drehendes_pendel_4}
    \end{subfigure}
    \caption{Hier stellen die orangenen Punkte die pro Iteration berechneten Momentanpole dar. Der grüne Graph steht jeweils für die Anzahl der X-Koordinaten der berechneten Momentanpole und der orangene Graph für die Y-Koordinaten. Die bläulichen Funktionen stellen jeweils die Gauß-Verteilungen dar.}
    \label{fig:drehendes_pendel_1_4}
\end{figure}

Anhand von Abbildung~\ref{fig:drehendes_rad_1_4} ist bereits abzusehen, dass die Standardabweichung erheblich geringer ist.
Die Ellipsen um die Erwartungswerte stellen die Vertrauensbereiche dar.
Jeder Kreis stellt hierbei ein vielfaches der Standardabweichung dar.
Die Standardabweichung reduziert sich für die X- und Y-Koordinate jeweils von $4.2$ und $4.7$ auf $3.0$ und $2.6$ nach einer kompletten Umdrehung des Gliedes. 

\subsection{Translatives Gestellglied}

Für ein sich translativ bewegendes Glied ist entsprechend zu erwarten, dass der Momentanpol sich im unendlichen befindet.
Die numerische Herangehensweise lässt hierbei also vermuten, dass die Varianz der berechnten Punkte sehr hoch ist.
Ein eindeutiges Ergebnis zu bekommen ist hier nicht zu erwarten.
Im Gegenteil lässt hier ein besonders schlechtes Ergebnis den Schluss zu, dass es sich entsprechend um ein sich translativ bewegendes Glied handelt.
Bei der Berechnung des Momentanpols des sich translativ bewegenden Gelenkes teilweise ein Winkel von in etwa $\pi$ ermittelt wird, weshalb in Abbildung~\ref{fig:translatives_glied_1} die Momentanpole teilweise auf dem Pfad des Gelenkes liegen.
Im Vergleich mit dem Ansatz welcher die absolute Geschwindigkeit über Lucas-Kanade und die Winkelgeschwindigkeit über die Regressionsgerade bestimmt wird dieser Fehler allerdings nicht weiter beachtet, da der letztere dem ersten offensichtlich überlegen ist.
Beim zweiten Ansatz wurde aus diesem Grund der eingeschlossene Winkel durch \lstinline{dw = dw > Math.PI / 2 ? dw - Math.PI : dw;} korrigiert, was jeweils den kleineren der beiden eingeschlossenen Winkel bevorzugt.
Der rote Punkt in Abbildung~\ref{fig:translatives_glied_3} stellt hier den verfolgten Punkt dar und nicht den Momentanpol.
Die Standardabweichung, sofern ein Vergleich an dieser Stelle sinnvoll ist, hat im ersten Versuch einen Wert jeweils um die $100$, während dieser im zweiten Ansatz nicht mehr berechenbar scheint\footnote{Die jeweils berechnete Standardabweichung wird hier als \lstinline{NaN} angegeben.}. % TODO warum?

\begin{figure}
    \centering
    \begin{subfigure}[t]{0.45\textwidth}
        \includegraphics[width=\textwidth]{gfx/translatives_glied_1.png}
        \caption{Versuch: \name{momentanpol3\_1.html}.}\label{fig:translatives_glied_1}
    \end{subfigure}
    \begin{subfigure}[t]{0.45\textwidth}
        \includegraphics[width=\textwidth]{gfx/translatives_glied_3.png}
        \caption{Versuch: \name{momentanpol3\_3.html}.}\label{fig:translatives_glied_3}
    \end{subfigure}
    \caption{Hier sind die Bewegungen eines sich translativ bewegenden Gelenkes nachvollzogen. Im zweiten Bild wird zudem die Gerade zwischen dem aktuellen Punkt und dem Momentanpol gezeichnet, welche hier in der Tat orthogonal zur Bewegungsrichtung steht. Auf dem Bild selber sollten keine Momentanpole zu finden sein.}
    \label{fig:translatives_glied_1_3}
\end{figure}

% \section{Nutzung des Lucas-Kanade Algorithmus}

% Es ist davon auszugehen, dass die vorangegangene Methode eine gewisse Ungenauigkeit mit sich bringt, da die Eingangsparameter bereits mit einer signifikanten Streuung eingehen.
% Hierbei ist zu vermuten, dass die Nutzung der Punkte mit größter Distanz die größere Fehlerquelle darstellt.
% Aus diesem Grund soll die Absolutgeschwindigkeit nicht über diese, sondern über die Punkte gemessen werden, welche durch den in Kapitel~\ref{ch:lucas_kanade} eingeührten Algorithmus ermittelt werden.

% Diese Herangehensweise benötigt nun jedoch eine Mischung der Funktionen welche vorher für die Berechnung der unterschiedlichen Bilder genutzt wurde.
% Für die Bestimmung der Regressionsgerade wird die Punktwolke genommen welche durch die \lstinline{PointCloud.fromImages} ermittelt wird.
% Die \lstinline{LucasKanade.step} Funktion benötigt jedoch ein \lstinline{cv.Mat} Objekt des Bildes.
% Deshalb wurde auf der \lstinline{Group} Klasse eine \lstinline{step} Funktion definiert.

% \begin{lstlisting}[language=JavaScript, caption={Definition der \lstinline{step} Funktion der \lstinline{Group} Klasse} \label{lst:group_step}]
% step(fn) {
%     this.lucasKanade();
%     const pts = this.regressionLine();

%     fn?.call(undefined, pts);
% }
% \end{lstlisting}

% In Listing~\ref{lst:group_step} wird beschrieben wie der Lucas-Kanade Ansatz zusammen mit der Regressionsgerade genutzt wird.
% In der \lstinline{lucasKanade} Funktion wird entsprechend das Bild von \lstinline{cnv1} als ein \lstinline{cv.Mat} Objekt ausgelesen und die \lstinline{LucasKanade.step} Funktion aufgerufen.
% Für diesen Test wird zunächst der \lstinline{maxCorners} für den \name{Shi-Tomasi} Algorithmus auf eins gesetzt, damit nur ein Punkt verfolgt wird.
% Der Speiche des Rades wurden ebensfalls zwei Punkte hinzugefügt, damit diese als eine Ecke von diesem erkannt werden können.
% Der dann verfolgte Punkt wird dem \lstinline{Group} Objekt hinzugefügt.

% Die weitergehende Berechnung ist die gleiche wie in Kapitel~\ref{ch:drehendesRad}.
% Die Standardabweichung des hier ermittelten Momentanpols ist hier viel kleiner.
% Dies ist auch dann der Fall, wenn direkt aufeinanderfolgende Iterationen genutzt werden.
% Erhöht man die Anzahl dieser, so reduziert sich die Standardabweichung weiter.
% Für das sich drehende Rad ist anzumerken, dass wenn der verfolgte Punkt sich im höchsten Punkt des von ihm gezeichneten Zykloiden befindet der Momentanpol dort falsch vorhergesagt wird.
% Das kann jedoch mit entsprechenden Maßnahmen in einer Nachbearbeitung entfernt werden.

% Für die Ermittlung des Momentanpols des Pendels sind die Ergebnisse sehr viel besser.
% Dort wird der Momentanpol mit einer Standardabweichung von unter eins korrekt vorhergesagt.
% Für das sich translativ bewegende Glied steigt die Standardabweichung Erwartungsgemäß in extrem Hohe werte, sodass auch diese als eine vielversprechende Beobachtung bezeichnet werden kann.

% TODO es kann noch mit den Mittelsenkrechten von den Punkte gearbeitet werden statt den Regressionsgeraden. Vielleicht sind die ja genauer.

\section{Beobachtungen}
Es hat sich gezeigt, dass die naive herangehensweise von den am weitest entfernten Punkten in Verbindung mit der Regressionsgerade durchaus Erwartungswerte liefert welche dem tatsächlichen Momentanpol entsprechen.
Werden jedoch lediglich Momentaufnahmen betrachtet, so sind diese wahrscheinlich nicht als gute Bezugspunkte für weitere Prognosen zur Zusammenstellung des Mechanismus geeignet.
Die Kombination aus der Ermittlung der Regressionsgeraden und der Verfolgung von Punkten durch den Lucas-Kanade Algorithmus haben jedoch eine sehr geringe Streuung.
Es kann also davon ausgegangen werden, dass auf diese Weise ermittelte Erwartungswerte nach nur wenigen Iterationen nutzbare Pole darstellen um einen Mechanismus zu rekonstruieren.

Bisher wurde die Ermittlung des Momentanpols eines isolierten Gliedes betrachtet.
Da ein Mechanismus zwangsläufig aus mehreren Gliedern besteht soll nun betrachtet werden wie ein mehrgliedriges Modell in seine Komponenten geteilt werden kann.
